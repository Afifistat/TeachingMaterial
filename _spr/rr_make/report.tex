\documentclass{article}

\usepackage{listings,graphicx}
\usepackage[a4paper,margin=2cm,noheadfoot]{geometry}
\begin{document}

\title{Example report compiled using Make}
\author{Stephen J. Eglen}
\date{\today}
\maketitle


\section{Introduction}

This is an example of using a \texttt{Makefile} to assemble all of the results
and graphs needed to re-generate a report.  

We can draw the graphical dependencies involved in regenerating all of
these files.

So, e.g. if I edit the \texttt{.tex} file, then only the report needs to be
regenerated, not all of the simulations.  (Compare this with Sweave.)


\section{Parameters}

Here is the input file that provides the key parameters for the
simulations:

{\lstinputlisting{params.R}}

\begin{figure}[h]
  \centering
  \includegraphics[width=8cm]{sim1.pdf}
  \includegraphics[width=8cm]{sim2.pdf}
  \caption{Result of simulation. Left: rnorm.  Right: runif.}
  \label{fig:res}
\end{figure}

\clearpage

%% Technically you could argue that I should add the following files
%% as dependencies to the Makefile entry for report.pdf, but they are 
%% included just to save paper, and would complicate the description
%% of the Makefile.

\section{simulator.R}

\lstinputlisting{simulator.R}

\section{plotter.R}
\lstinputlisting{plotter.R}

\section{Makefile}
\lstinputlisting{Makefile}

\end{document}
